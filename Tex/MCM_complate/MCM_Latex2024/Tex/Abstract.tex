
A traditional bathtub cannot be reheated by itself, so users have to add hot water from time to time. Our goal is to establish a model of the temperature of bath water in space and time. Then we are expected to propose an optimal strategy for users to keep the temperature even and close to initial temperature and decrease water consumption.
    
To simplify modeling process, we firstly assume there is no person in the bathtub. We regard the whole bathtub as a thermodynamic system and introduce heat transfer formulas.

We establish two sub-models: adding water constantly and discontinuously. As for the former sub-model, we define the mean temperature of bath water. Introducing Newton cooling formula, we determine the heat transfer capacity. After deriving the value of parameters, we deduce formulas to derive results and simulate the change of temperature field via CFD. As for the second sub-model, we define an iteration consisting of two process: heating and standby. According to energy conservation law, we obtain the relationship of time and total heat dissipating capacity. Then we determine the mass flow and the time of adding hot water. We also use CFD to simulate the temperature field in second sub-model.

In consideration of evaporation, we correct the results of sub-models referring to some scientists' studies. We define two evaluation criteria and compare the two sub-models. Adding water constantly is found to keep the temperature of bath water even and avoid wasting too much water, so it is recommended by us.

Then we determine the influence of some factors: radiation heat transfer, the shape and volume of the tub, the shape/volume/temperature/motions of the person, the bubbles made from bubble bath additives. We focus on the influence of those factors to heat transfer and then conduct sensitivity analysis. The results indicate smaller bathtub with less surface area, lighter personal mass, less motions and more bubbles will decrease heat transfer and save water.

Based on our model analysis and conclusions, we propose the optimal strategy for the user in a bathtub and explain the reason of uneven temperature throughout the bathtub. In addition, we make improvement for applying our model in real life.
\begin{keywords}
    Heat transfer, Thermodynamic system, CFD, Energy conservation
\end{keywords}
    
