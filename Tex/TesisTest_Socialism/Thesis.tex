%导言区
\documentclass[a4paper]{ctexart}
%宏包引用
\usepackage{graphicx} %图片调用
\usepackage{geometry} %设置页边距的包
\usepackage{pdfpages} %本将PDF文件加入到封面位置
\usepackage{fancyhdr} %页眉页脚
\usepackage{abstract} %摘要
\usepackage{fontspec} %导入字体包
\usepackage{setspace} %导入行距包
\usepackage{titletoc} %自定义目录分级标题
\usepackage{titlesec} %自定义分级章节标题

%页面设置
\geometry{left=2.5cm,right=2cm,top=2.54cm,bottom=2.54cm} %设置页边距
\pagestyle{fancy}
\fancyhf{}
\lhead{\heiti《社会主义发展史》课程期末论文}
\rhead{\heiti 研究马克思对资本主义的文化批判}
\rfoot{ \thepage} %页脚显示页码

%摘要设置
\renewcommand{\abstractnamefont}{\huge\heiti}

%对目录格式进行修改
%\titlecontents{level}[leftmargin]{before-code}{label-format}{separator-code}{after-code}
\renewcommand{\contentsname}{\Huge\heiti 目录}
\titlecontents{section}[1em]{\bfseries\large \vspace{7pt}}{\contentslabel{1em}\large}{\hspace*{-1em}}{~\titlerule*[0.6pc]{$.$}~\contentspage}
\titlecontents{subsection}[3em]{\vspace{6pt}}{\contentslabel{1.5em}}{\hspace*{-4em}}{~\titlerule*[0.6pc]{$.$}~\contentspage}
\titlecontents{subsubsection}[5em]{\vspace{5pt}}{\contentslabel{2.3em}}{\hspace*{-4em}}{~\titlerule*[0.6pc]{$.$}~\contentspage}


%正文区
\begin{document}

%封面
\begin{titlepage}
  \includepdf[pages={1}]{figures/cover.pdf} %外界自建封面,然后调用
\end{titlepage}

%摘要
\begin{abstract}
  \songti
  \fontsize{12pt}{17pt}\selectfont
  \thispagestyle{empty}
  马克思对资本主义的文化批判,作为马克思文化观的核心组成部分,是马克思主义
理论体系中的重要内容。这一思想不仅从文化角度深刻剖析了资本主义社会的内在
矛盾,还为我们理解资本主义文化提供了独特的视角。在本文中,我们将基于对资
本主义文化等相关概念的清晰界定,系统探讨马克思资本主义文化批判思想的产生
背景、发展历程、核心内容、显著特征及其对当代社会的深远影响。

马克思对资本主义的文化批判,是在资本主义经济快速发展与无产阶级革命运动蓬勃
兴起的背景下孕育而生的。他批判性地吸收了包括意大利人文精神在内的多种思想理论
精华,形成了自己独特的资本主义文化批判体系。这一思想体系经历了从早期到成熟的
不同阶段,包括“两个转变”时期、“新世界观”时期,以及马克思晚年的深化发展。在这
个过程中,马克思对资本主义文化的批判逐渐深入,最终形成了完整的历史唯物主义文化观。

其思想内容博大精深,对资产阶级法哲学、宗教文化、经济伦理及文化价值理念等均进
行了深刻批判。马克思不仅揭示了资本主义文化的本质和意识形态的虚伪性,还指出了
其历史进步性与阶级局限性。这种批判体现了唯物主义与辩证法的统一,科学精神与人
民立场的统一,以及理论批判与实践批判的统一
% 马克思对资本主义的文化批判内容极为丰富。他首先对资产阶级法哲学文化进行了
% 批判,揭示了法哲学的形而上学本质;其次,他对宗教文化进行了深入的剖析,揭开
% 了宗教的神秘面纱,并指出宗教批判是其他一切批判的前提;再次,他对资产阶级经
% 济伦理思想进行了批判,揭示了资本主义生产方式的奴役性和异化性,从经济根源上
% 揭示了资本主义道德的本质;最后,他对资本主义文化价值理念进行了剖析,厘清了
% 资本主义文化的价值诉求。

% 马克思对资本主义的文化批判具有鲜明的特征。首先,它体现了唯物主义与辩证法
% 的统一,将唯物主义的世界观与辩证法的方法论相结合;其次,它体现了科学精神与
% 人民价值立场的统一,坚持科学原则与人民利益的高度一致;最后,它体现了理论批
% 判与实践批判的统一,既在理论上进行深刻的剖析,又在实践中进行积极的斗争。

马克思对资本主义的文化批判对当代社会具有重要的意义。它为社会主义文化建设
提供了宝贵的理论资源和思想指导。在当前复杂多变的国际形势下,研究马克思的资
本主义文化批判思想有助于我们坚定中国特色社会主义的文化自信,加强主流意识形
态建设,抵御西方社会思潮的文化渗透。同时,这也为我们在新时代中国特色社会主
义文化建设中立足生活实践、坚持问题导向、坚守人民中心立场、弘扬民族文化提供
了重要的启示。\\[5pt]

{
    \raggedright\heiti\fontsize{13pt}{17pt}\selectfont 
    关键词:马克思、资本主义、文化批判
    }
\end{abstract}
\newpage

%目录
\begin{center}
  \tableofcontents
\end{center}
\thispagestyle{empty} 
\newpage

% 正文格式设置:
%正文字体设置
\setmainfont{SimSun} % 设置正文中文字体为宋体
\fontsize{12pt}{18pt}\selectfont % 设置正文字号为小四号,行距为20磅
%正文分级格式:
% 自定义\section格式  
\titleformat{\section}
{\normalfont\LARGE\heiti\centering} % 字体、大小、加粗、颜色、缩进、排版
{\thesection} % 编号格式  
{1em} % 编号与标题内容的水平间距  
{} % 编号后的额外格式(此处为空)  

% 自定义\subsection格式  
\titleformat{\subsection}
{\normalfont\Large\heiti} % 字体、大小、斜体、颜色、缩进、排版
{\thesubsection} % 编号格式,注意这里通常不需要点号  
{1em} % 编号与标题内容的水平间距  
{} % 编号后的额外格式(此处为空) 

% 自定义\subsubsection格式  
\titleformat{\subsubsection}
{\normalfont\fontsize{13pt}{17pt}\selectfont \heiti\setlength{\leftskip}{1em}} % 字体、大小、斜体、颜色、缩进、排版 
{\thesubsubsection} % 编号格式,注意这里通常不需要点号  
{1em} % 编号与标题内容的水平间距  
{} % 编号后的额外格式(此处为空)   

%正文开始
\setcounter{page}{1}
\input{Tex/Text.tex}
\newpage

%参考文献
% \section*{参考文献}  %手动不引用文献
\addcontentsline{toc}{section}{参考文献}
% \pagestyle{empty}
\fontsize{10pt}{13pt}\selectfont
% \raggedright
\begin{thebibliography}{99}%引用上限
  \bibitem{bib1}胡海波,郭风志马克思恩格斯文化观研究[M].北京:中国书籍出版社,2013:111--113.
  \bibitem{bib2}房广顺.马克思和恩格斯的资本主义观及其当代发展[J].理论月刊,2014(10):5.
  \bibitem{bib3}张三元.论资本逻辑与现代性文化[J].江汉论坛,2019(03):71-77.
  \bibitem{bib4}宁德业.《共产党宣言》的文化思想及其当代价值[J].当代世界与社会主义,2018(01):61.
  \bibitem{bib5}相雅芳,尚天钰.文化自信何以可能:马克思对资本主义文化的三重批判及当代价值[J].兰州大学学报(社会科学 版),2022,50(0 1):49.
  \bibitem{bib6}胡海波.郭凤志.马克思恩格斯文化观研究[M].北京:中国书籍出版社,2013:114—116.
  \bibitem{bib7}胡芳.沦我国摆脱资本主义文化危害的可能性与困境[J].理论月刊,2012(09):94.
  \bibitem{bib8}姜迎春.论马克思的文化批判思想及其当代价值——兼论中国特色社会主义文化建设的理论基础[J].毛泽东邓小平理论 研究,2009(07):13.
  \bibitem{bib9}侯衍社.资本主义文化价值观二重性剖析[J].北京行政学院学报,2011(3):59-61.
  \bibitem{bib10}[德]马克斯·韦伯.新教伦理与资本主义精神[M].于晓,陈维纲译,北京:三联书店,1987:8.
  \bibitem{bib11}[美]丹尼尔呗尔.资本主义文化矛盾[M].赵一凡,蒲隆,任晓晋译.北京:三联书店,1989:100.
  \bibitem{bib12}马克思恩格斯选集(第1卷)[M].北京:人民出版社,2012.227.
  \bibitem{bib13}习近平.高举中国特色社会主义伟大旗帜\quad 为全面建设社会主义现代化国家而团结奋斗------在中国共产党第二十次全国代表大会上的报告[M].北京:人民出版社,2022:42-43.
  \bibitem{bib14}中共中央宣传部.习近平新时代中国特色社会主义思想三十讲[M].北京:学习出版社,2019:203.
\end{thebibliography}
\end{document}